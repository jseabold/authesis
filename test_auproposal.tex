%Last modified: 2013 Jul 19
\documentclass[12pt]{article}
\usepackage{auproposal}
\usepackage{url}
% IMPORTANT NOTES:
% 1) You MUST run LaTeX THREE times after runing BibTeX.
% 2) Make sure that you have the LATEST version of the AUTHESIS class files
%    before you hand in the final version of your thesis or dissertation.

% \usepackage commands should go before the \begin{document}

%\usepackage{ctable} %You should learn about ctable!!

% IF you use the 'cas' option instead of the preferred 'econ' option,
% you MAY uncomment the following line.  (NOT recommended.)
%\usepackage{ulem}

\usepackage{natbib}  % natbib citation style

\begin{document}

% Declarations for Front Matter

%
% IMPORTANT: IF A TITLE OR SUBTITLE EXCEEDS MORE THAN ONE LINE,
%            THERE SHOULD ONLY BE  48 CHARACTERS PER LINE.
%
% The dissertation title must be capitalized for AU-CAS
\title{MY THESIS OR DISSERTATION TITLE IN CAPITALS \\
(WITH 48 CHARACTERS OR FEWER PER LINE)\\
AS AN INVERTED PYRAMID}

\author{Will E. Finnish}
\degreeyear{2525}
\degree{Doctor of Philosophy}
\chair{Professor Knowit Icann}
\secondreader{Professor Ivory Tower}
\thirdreader{Professor Mih Sing Cite}
% If you have more than a chair and two readers, you need to
% edit file AUTHESIS.CLS and add a line to the table in the
% coverpage. I have defined up to sixthreader (that is, seven
% all told).

\degreefield{Introspective Empiricism}

% The following command makes the title page, it is duplicated to produce
% two copies of cover page, as required by AU-CAS.

\maketitle


% The frontmatter environment uses roman lower case page numbering. The
% abstract, acknowledgements, table of contents, list of figures, and
% list of tables are a part of this environment.

\begin{frontmatter}

% The Guide previously two abstract pages, one numbered, one without numbers.
% Now only a numbered abstract page is used in the thesis.
% The 'abstract' macro generates an unnumbered abstract page for the proposal.

\begin{abstract}
This is an abstract.  There are no page numbers.
The \emph{Guide} says an abstract should not exceed 350 words.

This is an abstract. This is an abstract. This is an abstract. This is an abstract. This is an abstract. This is an abstract. This is an abstract. This is an abstract. This is an abstract. This is an abstract. This is an abstract. This is an abstract. This is an abstract. This is an abstract. This is an abstract. This is an abstract. This is an abstract. This is an abstract. This is an abstract. This is an abstract. This is an abstract. This is an abstract. This is an abstract. This is an abstract. 
\end{abstract}

\newpage



\tableofcontents

\newpage

\listoftables

\newpage

\listoffigures

\newpage


\end{frontmatter}


\section{Introduction}

Every proposal should have an introduction.
The introduction should introduce the concepts, background, and goals of the dissertation.

Another paragraph.
Another sentence. Another sentence. Another sentence. Another sentence. Another sentence. Another sentence. Another sentence. Another sentence. Another sentence. Another sentence. Another sentence. Another sentence. Another sentence. Another sentence.%
\footnote{%
Here is a footnote.}
%





\subsection{First subsection}

Another paragraph.
Another sentence. Another sentence. Another sentence. Another sentence. Another sentence. Another sentence. Another sentence. Another sentence. Another sentence. Another sentence. Another sentence. Another sentence. Another sentence. Another sentence. 

% Captions for tables must go above the table.
\begin{table}[p]\centering
\caption[Short Caption]{A Table}
\begin{tabular}{lr}\hline\hline
Title & Author \\ \hline
War And Peace & Leo Tolstoy \\
The Great Gatsby & F. Scott Fitzgerald \\ \hline
\end{tabular}
\end{table}


\subsubsection{Subsubsection for test purposes}

Another paragraph.
Another sentence. Another sentence. Another sentence. Another sentence. Another sentence. Another sentence. Another sentence. Another sentence. Another sentence. Another sentence. Another sentence. Another sentence. Another sentence. Another sentence. 



\section{Previous Work}

Some other research was once performed.


\subsection{Section}
%first section

Some was good and some was bad.

% Captions for figures must go below the figure.  
\begin{figure}
\centering MY FIRST FIGURE
\caption{Figure caption must be below figure.}
\end{figure}


\subsection{Subsection}

Some was neither good or bad.

\begin{figure}
\centering MY SECOND FIGURE
\caption{Second figure.}
\end{figure}


\subsubsection{Subsubsection}

Surely mine will be better.
Here is a bulleted list of reasons why.
\begin{itemize}
\item Reason 1.
Some people hate the way test wraps in our bulleted lists,
but that is what the Guide says we have to do---apparently based on Turabian.
You can argue with CAS if you wish \dots
\item Reason 2.
\item Reason 3.
\item Reason 4.
\end{itemize}


\section{Projected Schedule of Deliverables}


\subsection{Section}


\begin{figure}
\centering MY THIRD FIGURE
\caption{First figure in second section.}
\end{figure}



\subsection{Subsection}

\begin{figure}
\centering MY FOURTH FIGURE
\caption{Second figure in second section.}
\end{figure}


\subsubsection{Subsubsection}


\section{Section}

\begin{figure}
\centering MY FIFTH FIGURE
\caption{First figure in third section.}
\end{figure}


\subsection{Subsection}

\section{Section}
\subsection{Subsection}
\subsection{Subsection}
\section{Section}
\subsection{Subsection}
\subsection{Subsection}

\begin{figure}
\centering MY SIXTH FIGURE
\caption{Second figure in third section.}
\end{figure}


\subsubsection{Subsubsection}


\section{Conclusion}

All is well that ends.

This will end soon. This will end soon. This will end soon. This will end soon. This will end soon. This will end soon. This will end soon. This will end soon. This will end soon. This will end soon. This will end soon. This will end soon. This will end soon. This will end soon. This will end soon. This will end soon. This will end soon. This will end soon. This will end soon. This will end soon. This will end soon. This will end soon. This will end soon. This will end soon. This will end soon. This will end soon. This will end soon. This will end soon. This will end soon. This will end soon. This will end soon. This will end soon. This will end soon. This will end soon. This will end soon. This will end soon. This will end soon. This will end soon. This will end soon. This will end soon. This will end soon. This will end soon. This will end soon. This will end soon. This will end soon. This will end soon. This will end soon. This will end soon. This will end soon. This will end soon. 

This will end soon. This will end soon. This will end soon. This will end soon. This will end soon. This will end soon. This will end soon. This will end soon. This will end soon. This will end soon. This will end soon. This will end soon. This will end soon. This will end soon. This will end soon. This will end soon. This will end soon. This will end soon. This will end soon. This will end soon. This will end soon. This will end soon. This will end soon. This will end soon. This will end soon. This will end soon. This will end soon. This will end soon. This will end soon. This will end soon. This will end soon. This will end soon. This will end soon. This will end soon. This will end soon. This will end soon. This will end soon. This will end soon. This will end soon. This will end soon. This will end soon. This will end soon. This will end soon. This will end soon. This will end soon. This will end soon. This will end soon. This will end soon. This will end soon. This will end soon. 

See?

%IMPORTANT: use the \appendix command to change numbering to lettering
\appendix

\section{Some Ancillary Stuff}


\section{Some More\\Ancillary Stuff}



\ssp                           %IMPORTANT: set single spacing for bibliography
\nocite{*}
\bibliographystyle{au-cms}     % you must download au-cms.bst to run this test
\bibliography{autest}          % you must download autest.bib to run this test

\end{document}


